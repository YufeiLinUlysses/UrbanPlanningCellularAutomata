 \documentclass[a4paper,12pt]{report}

%Packages Used
\usepackage{amssymb,latexsym,amsmath}     % Standard packages
\usepackage{setspace}
\usepackage{sectsty}
\usepackage{titlesec}
\usepackage{hyperref}
\usepackage{bookmark}
\usepackage{graphics,graphicx}
\usepackage{tikz}
\usepackage{mathtools}
\usepackage{graphicx}
\usepackage{amssymb}
\usepackage{esvect}
\usepackage{chngcntr}

\DeclarePairedDelimiter\abs{\lvert}{\rvert}%
\DeclarePairedDelimiter\norm{\lVert}{\rVert}%


\bookmarksetup{
  numbered,
  open
}
\renewcommand*{\thesection}{\arabic{section}}
\onehalfspacing

\newcommand{\suchthat}{\;\ifnum\currentgrouptype=16 \middle\fi|\;}

%Margins
\addtolength{\textwidth}{1.0in}
\addtolength{\textheight}{1.00in}
\addtolength{\evensidemargin}{-0.75in}
\addtolength{\oddsidemargin}{-0.75in}
\addtolength{\topmargin}{-.50in}

%%%%%%%%%%%%%%%%%%%%%%%%%%%%%% 
% Theorem/Proof Environments %
%%%%%%%%%%%%%%%%%%%%%%%%%%%%%%
\newtheorem{theorem}{Theorem}
\newenvironment{proof}{\noindent{\bf Proof:}}{$\hfill \Box$ \vspace{10pt}}
\sectionfont{\fontsize{12}{15}\selectfont}
\titlespacing*{\section}{0.5pt}{0.25\baselineskip}{0.25\baselineskip}

\counterwithin*{equation}{section}
\counterwithin*{equation}{subsection}

\usepackage{xcolor}
\usepackage{titlesec}
\titleformat{\section}[block]{\bfseries\filcenter}{}{1em}{}
\titleformat{\subsection}[hang]{\bfseries}{}{1em}{}
\setcounter{secnumdepth}{0}


\begin{document}
\noindent
Yufei Lin

\noindent
Final Essay

\noindent
Apr \(30^{th}\) 2020

\begin{center}
\textbf{Final Essay}
\end{center}

\section{Abstract}

This research seeks to improve the existing cellular automata models for simulating sustainable city development. The existing literature has taken land use dynamics as a direct representation of city development. This literature has accounted for the economic, environmental, and social factors affecting land use transition probabilities within a context of a growing city. We will expand upon this model to investigate how changes to the transition probabilities could influence the way a city develops. 

\section{Introduction}

In this research, we started building out a simple land simulation system with four different land use types: Nature, Residential, Commercial and Industrial, where all cells change follows a specific transition matrix. Then, we expand this model to look at how neighbouring cells would effect the transition probabilities of an existing cell by first looking at a simpler model where we only take in the different number of land types into account, and then look at a combination of different number of land use types and a set of given transition matrices. And, we look at the percentage of nature and residential, and the average distance from a cell to all three other types of land uses.  



\end{document}